\documentclass{tufte-handout}
\usepackage{amsmath,stmaryrd,amssymb,amsthm,url,booktabs,hyperref}

\usepackage{euler}



\title{The diagonal argument redux\thanks{This document is released \\ under a CC0 license: \href{http://creativecommons.org/publicdomain/zero/1.0/}{\texttt{creativecommons.org/publicdomain/zero/1.0/}}}}
\author{David Michael Roberts}
\date{\today}


\theoremstyle{definition}
\newtheorem{prop}{Proposition}
\newtheorem{lemma}{Lemma}
\newtheorem*{definition}{Definition}
\newtheorem{example}{Example}
\newtheorem*{theorem}{Theorem}
\newtheorem{corollary}{Corollary}
\newtheorem{q}{Question}
\newtheorem*{conj}{Conjecture}
\newtheorem*{rem}{Remark}

\DeclareMathOperator{\id}{id}
\DeclareMathOperator{\pr}{pr}
\DeclareMathOperator{\Pre}{Pre}
\def\CC{\mathcal{C}}

\begin{document}
\maketitle

\begin{abstract}
The following is a generalisation of the contrapositive to Lawvere's fixed-point theorem, from categories with finite products to a much more general setting.\end{abstract}

Lawvere's fixed-point theorem\footnote{F.W.~Lawvere, \emph{Diagonal arguments and cartesian closed categories}, Lecture Notes in Mathematics, \textbf{92} (1969), 134--45. Republished in:
Repr.\ Theory Appl.\ Categ., No. 15 (2006), pp 1--13.} is a result about cartesian closed categories whose contrapositive is an abstract, categorical version of Cantor's diagonal argument.
It says that if $A \to Y^A$ is surjective on global points\sidenote{A global point of $A$ is a map $1\to A$}, then every endomorphism of $Y$ has a fixed global point.
However, Lawvere goes further and observes that the proof of his theorem does not really use the \emph{closed} structure except to translate from the theorem statement to the setting in which the fixed point is constructed.
Thus with appropriate unwinding of the defintions involved, the theorem makes sense in an arbitrary category with finite products: a \emph{cartesian} category.

On the face of it, this seems like the ultimate version of the diagonal argument: one needs the terminal object (a $0$-ary product) to talk about global points, and the binary products to talk about the function $A\times A \to Y$ to which diagonalisation is applied.
However, closer inspection of the proof makes it clear that it never uses the projection maps that characterise cartesian categories among monoidal categories.
Further, the universal property of the product is never used aside from constructing the diagonal $A\to A\times A$. 
Monoidal categories with diagonal maps exist, and so one can see immediately that Lawvere's proof in the cartesian case applies to more general monoidal categories with diagonals.
A second, even closer, inspection shows that the coherence structure of the monoidal category is never used.
Ultimately, while the first instinct is to replace the terminal object with the monoidal unit, we can remove the reliance on global points entirely, and just use arbitrary generalised elements.
We can thus supply what seems like the minimal structure that supports a diagonal argument, which is what Lawvere ultimately uses his fixed-point theorem for.


\begin{definition}
A \emph{magmoidal category}\footnote{A.~Davydov, \emph{Nuclei of categories with tensor products}, Theory
Appl.\ Categ., Vol. 18, 2007, No. 16, pp 440--472.} is a category $\CC$ equipped with a functor $\# \colon \CC\times \CC\to \CC$.
A magmoidal category is said to have \emph{diagonals} if there is a natural transformation $\delta\colon \id_C \Rightarrow \#\circ \Delta_\CC$. 
\end{definition}


Examples include categories with finite products (\emph{cartesian} magmoidal categories), and monoidal and skew-monoidal categories with diagonals.
For a nontrivial example, consider a (skew-)monoidal category $(\mathcal{C},\otimes)$ with finite coproducts. 
Then for each object $x\in \mathcal{C}$, $a\# b := c\amalg (a\otimes b)$ defines the structure of a magmoidal category.
If moreover the (skew-)monoidal structure admits diagonals, then so does this magmoidal structure.
For the following we shall fix a magmoidal category with diagonals $(\CC,\#,\delta)$.


In the cartesian setting, a map $A\times B \to C$ can be viewed as an $A$-parameterised family of maps $B\to C$; in the cartesian closed case this is of course equivalent to a map $A \to C^B$.
In our setting of a magmoidal category, we still want to think of a morphism $A\# B \to C$ as being an $A$-parametrised family of maps $B\to C$.

\medskip

\hrule
\begin{quote}
\small How does one see this? Given any $a\colon X\to A$, hence supposedly picking out one such map, then for any $Y\xrightarrow{p} X \in \mathcal{C}_{/X}$ and $b\colon Y\to B$ we get 
\[
F(p^*a,b)\colon Y\xrightarrow{\delta} Y\# Y \xrightarrow{p\# id_Y} X\# Y \xrightarrow{a\# b} A \# B \xrightarrow{F} C
\] 
So, if we think of $a$ as being an element of $A$ at stage $X$, then for any later stage $Y$ (via $p$), we have an assignment of elements $b\mapsto F(p^*a,b)$.
More formally, recall that for any object $D$ of $\mathcal{C}$, there is a presheaf on $\mathcal{C}_{/X}$ given by 
\[
  X^*y_D\colon (Y\xrightarrow{p} X) \mapsto \mathcal{C}(Y,D).
\] 
% Here $y_D$ is the reprentable presheaf arising from $D$ and $X^*\colon \Pre(\mathcal{C}) \to \Pre(\mathcal{C}_{/X})$.
Given $a$ as above, we get a map of presheaves $F_a\colon X^*y_B \to X^*y_C$, with component $(Y\xrightarrow{p} X)  \mapsto \left(b\mapsto F(p^*a,b)\right)$.\marginnote{Here $p^*a = a\circ p$.}
\end{quote}


\hrule

\medskip

We can ask whether, for a given $A$, one can find \emph{every} map $B\to C$ inside some given $A$-parametrised family $A\# B \to C$.

\begin{definition}
In any magmoidal category with diagonals $(C,\#,\delta)$, a map $F\colon A\#B\to C$ is an \emph{incomplete parametrisation} of maps $B\to C$ if there exists an $f\colon B\to C$ such that for all $a\colon X\to A$, there is $Y\xrightarrow{p}X$ and $b = b_a\colon Y\to B$ with $f\circ b \not= F\circ((p^*a)\# b)\circ \delta\colon X\to C$.
\end{definition}

\noindent
Thinking of $F$ as an $A$-parametrised family of maps as above, this says that there is some $f\colon B\to C$ such that for no $a\colon X\to A$ is $X^*y_f = F_a\colon X^*y_B\to X^*y_C$.
Finally, let us say that an endomorphism $\sigma\colon C\to C$ in a category is \emph{free} if for all $c\colon X\to C$, $\sigma\circ c \not=c$.

\begin{theorem}
  For $(C,\#,\delta)$ a magmoidal category with diagonals, and $\sigma \colon C\to C$ a free endomorphism, every $F\colon A\# A \to C$ is an incomplete parametrisation of maps $A\to C$.
\end{theorem}

\begin{proof}
  Define $f = \sigma\circ F\circ \delta$ and take $p=\id_X$. 
  Then for all $a\colon X\to A$, we have $f\circ a = \sigma \circ F\circ \delta \circ a= \sigma \circ F \circ (a\#a)\circ \delta \not=F\circ (a\#a)\circ \delta$.
\end{proof}

\noindent
This covers all the cases of the diagonal argument by Lawvere, but as mentioned above it no longer relies on global points $1\to A$ etc as in Lawvere's presentation, but uses arbitrary generalised elements. The question remains: is this the contrapositive to a fixed-point theorem generalising Lawvere's?







\end{document}