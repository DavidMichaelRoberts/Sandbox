\documentclass{amsart}
\usepackage{xcolor}
\usepackage{hyperref}
\usepackage[cmtip,all,color]{xy}\UseCrayolaColors

\begin{document}
\title{A braided monoidal category is symmetric if and only if the product is braided}
\author{David Michael Roberts}
\date{17 Dec 2019}

\maketitle

\noindent
\textbf{The following diagram commutes,} by virtue of the symmetric monoidal category axioms, and shows in that case $\otimes\colon \mathcal{C} \times \mathcal{C} \to \mathcal{C}$ is a \emph{braided} strong monoidal functor. That $\otimes$ is a \emph{strong} monoidal functor is due to Joyal--Street (`Braided monoidal categories'\footnote{Available at \url{http://web.science.mq.edu.au/~street/JS1.pdf}}, Proposition 2). The vertical composites down the sides below are the comparison maps for pre- and post-$\otimes$ products given by Joyal--Street.

\[\centerline{
	\xymatrix{
	 & (AB)(CD) \ar@[orange][rrr]^\sigma \ar@[blue][dl]_\alpha
	&&&
	\ar@[blue][dr]^\alpha (CD)(AB) \ar@[blue][d]^{\alpha^{-1}}& 
	\\
	%
	%
	A(B(CD))\ar@[blue][d]_{A\alpha^{-1}} \ar@[orange][dr]^{A\sigma}& 
	&
	& 
	&
	((CD)A)B \ar@[blue][d]^{\alpha B}  & 
	C(D(AB)) \ar@[blue][d]^{C\alpha^{-1}}\\
	%
	%
	A((BC)D)\ar@[orange][d]_{A(\sigma D)}&
	A((CD)B)\ar@[blue][d]^{A\alpha} \ar@[blue][rr]^{\alpha^{-1}}&&
	(A(CD))B\ar@[blue][d]_{\alpha^{-1}B} \ar@[orange][ur]^{\sigma B}& 
	(C(DA))B \ar@[orange]@<1.5ex>[d]^{(C\sigma)B} 
		\ar@[orange]@<-1.1ex>[d]^{\color{red}\stackrel{\ast}{=}}_{(C\sigma^{-1})B} 
		\ar@[blue][r]^\alpha &
	C((DA)B) \ar@[orange][d]^{C(\sigma B)}\\
	%
	%
	A((CB)D)\ar@[blue][d]_{A\alpha}&
	A(C(DB))\ar@[blue][dr]^{\alpha^{-1}}&&
	((AC)D)B \ar@[blue][dl]_\alpha \ar@[orange][dr]^{(\sigma D)B}& 
	(C(AD))B \ar@[blue][r]^\alpha& C((AD)B)\ar@[blue][d]^{C\alpha}\\
	%
	%
	A(C(BD))\ar@[blue][dr]_{\alpha^{-1}}\ar@[orange][ur]^{A(C\sigma)}&
	 & (AC)(DB)\ar@[orange][drr]_{\sigma(DB)} & 
	 &
	((CA)D)B \ar@[blue][d]^\alpha \ar@[blue][u]_{\alpha B}& 
	C(A(DB))\ar@[blue][dl]^{\alpha^{-1}}\\
	%
	%
	 &(AC)(BD)\ar@[orange][rrr]_{\sigma \sigma} \ar@[orange][ur]_{(AC)\sigma} 
	&&&(CA)(DB)&  
	}
	}
\]
% \medskip

\noindent
In the diagram I have written $AB:= A\otimes B$ etc, and identity arrows $1_X$ are written as just $X$. 
Isomorphisms arising from associators ($\alpha$) are in blue, those arising from the braiding ($\sigma$) are in orange.
Notice that the equality $\color{red}\stackrel{\ast}{=}$ is the only place where the symmetry axiom is used.

\medskip
\noindent Conversely, take $\mathcal{C}$ to be merely braided and $\otimes$ a braided functor. If $B=C=I$ the diagram above implies $(I\sigma^{-1})I = (I\sigma)I\colon (I(DA))I \to (I(AD))I$. Applying naturality squares for $IX\to X$ and $XI\to X$ gives us that $\sigma^{-1}=\sigma$, hence $\mathcal{C}$ is symmetric.
\medskip

\noindent
I'm sure this result is rather old, but haven't found a reference. It follows from high-powered abstract machinery about $E_\infty$-algebras in $\mathbf{Cat}$, but I think this was possible to prove by the time of Mac~Lane's 1963 coherence theorem.


\end{document}